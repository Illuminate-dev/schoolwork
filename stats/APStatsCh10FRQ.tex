%&latex
% arara: latexmk: { options: [ '-pdf' ] }
% arara: latexmk: { options: ['-c' ] }

\documentclass{article}
\usepackage{amsmath}

\title{AP Stats Chapter 10 FRQ}
\author{Henry Beveridge}
\date{March 21, 2025}

\begin{document}

\maketitle

\section*{11.}

\subsection*{a)}

\textbf{STATE:} 99\% confidence interval for $p_1-p_2$, where $p_1$ is the proportion of prostate cancer diagnosed men assigned to the surgery that survived 
for 5 years and $p_2$ is the proportion of prostate-cancer diagnosed men treated with observation only that survived for 5 years.

\noindent\textbf{PLAN:} two-sample z test for $p_1 - p_2$

    \textbf{Random:} participants were randomly assigned

    \textbf{10\%:} likely 367 is more than 10\% of all prostate-cancer patients and thus same with 364.

    \textbf{Large Counts:} $\hat{p}_1n_1 = 245 \ge 10$, $(1-\hat{p}_1)n_1 = 122 \ge 10$, $\hat{p}_2n_2 = 223 \ge 10$ and $(1-\hat{p}_2)n_2 = 141 \ge 10$

\noindent\textbf{DO:}
\begin{align*}
  \hat{p}_1 - \hat{p}_2 &\pm z^* \sqrt{\frac{\hat{p}_1(1-\hat{p}_1)}{n_1}+\frac{\hat{p}_2(1-\hat{p}_2)}{n_2}} \\
  & \text{where } \hat{p}_1 = 0.668 \text{ and } \hat{p}_2 = 0.613 \text{ and} \\
  & n_1 = 367 \text{ and } n_2 = 364 \text{ and} \\
  & t^* = -\text{invNorm}(area:0.005) \approx 2.576 \\
  0.055 &\pm 0.091 \\
\end{align*}

\noindent\textbf{CONCLUDE:} Thus, we are 99\% confident that the true difference in proportions
of prostate cancer diagnosed men who are treated with surgery and with observation is between -0.036 and 0.146.

\subsection*{b)}
% Does the confidence interval provide convincing evidence that the true proportions differ? Justify your answer.

No, the confidence interval does not provide convincing evidence that the true proportions differ because the interval contains 0.

\section*{12.}

\subsection*{a)}
% Construct and interpret a 99% confidence interval for the difference in mean amount of food Piper eats
% when she is offered Pick-a-Pair and when she is offered Pickled Peppers.

\textbf{STATE:} 99\% confidence interval for $\mu_1 - \mu_2$, where $\mu_1$ is the mean amount of food Piper eats when she is offered Pick-a-Pair and $\mu_2$ is the mean amount of food Piper eats when she is offered Pickled Peppers.

\noindent\textbf{PLAN:} two-sample t test for $\mu_1 - \mu_2$

  \textbf{Random:} each trial is randomly assigned

  \textbf{10\%:} 31 is likely less than 10\% of all times Piper ate, and same with 30.

  \textbf{Large Counts:} $n_1 = 31 \ge 30$ and $n_2 = 30 \ge 30$

\noindent\textbf{DO:}
\begin{align*}
  \bar{x}_1 - \bar{x}_2 &\pm t^* \sqrt{\frac{s_1^2}{n_1}+\frac{s_2^2}{n_2}} \\ 
  & \text{where } \bar{x}_1 = 85.2 \text { and } \bar{x}_2 = 82.1 \text{ and} \\
  & s_1 = 3.45 \text{ and } s_2 = 4.62 \text{ and} \\
  & n_1 = 31 \text{ and } n_2 = 30 \text{ and} \\
  & t^* = -\text{invT}(area:0.005, df: 29) \approx 2.76 \\
  3.1 &\pm 2.89
\end{align*}

\noindent\textbf{CONCLUDE:} Thus, we are 99\% confident that the true difference in means of the amount of food Piper eats when she is offered Pick-a-Pair and when she is offered Pickled Peppers is between 0.21 and 5.99.

\subsection*{b)}
% Suppose we want to test the hypothesis that the mean amount of Pick-a-Pair that Pickles eats is higher 
% than the mean amount of Pickled Peppers she eats. State the null and alternative hypotheses for this test.

\textbf{Null Hypothesis:} $H_0: \mu_1 - \mu_2 = 0$ (the mean amount of food Piper eats when she is offered Pick-a-Pair is equal to the mean amount of food Piper eats when she is offered Pickled Peppers)

\noindent\textbf{Alternative Hypothesis:} $H_a: \mu_1 - \mu_2 > 0$ (the mean amount of food Piper eats when she is offered Pick-a-Pair is greater than the mean amount of food Piper eats when she is offered Pickled Peppers)

\subsection*{c)}
% The standardized test statistic is t = 2.96 and the P-value is 0.002. Draw an appropriate conclusion using
% α = 0.01.
Since the P-value (0.002) is less than the significance level (0.01), we reject the null hypothesis and
conclude that there is convincing evidence that the mean amount of food Piper eats when she is offered 
Pick-a-Pair is greater than the mean amount of food Piper eats when she is offered Pickled Peppers)

\section*{13.}

% Tai Chi is often recommended as a way to improve balance and flexibility in the elderly. Below are before-
% and-after flexibility ratings (on a 1 to 10 scale, 10 being most flexible) for a random sample of 8 men in
% their 80’s who took Tai Chi lessons for six months.
% Flexibility rating after Tai Chi 2 4 3 3 3 4 5 10
% Flexibility rating before Tai Chi 1 2 1 2 1 4 2 10

\subsection*{a)}
% (a) Explain why these are paired data.
These are paired data because the person that is being evaluated si the same person before and after the treatment,
so we have are comparing the same individual.

\subsection*{b)}
% Calculate and interpret the mean difference.

\textbf{Mean Difference:} $\frac{1+2+2+1+2+0+3+0}{8} = 11/8 = 1.375$

This means that the average flexibility rating after Tai Chi is 1.375 points higher than the average flexibility 
rating before Tai Chi for this sample.

\subsection*{c)}
% Researchers would like to know if the true mean difference (after – before) in flexibility rating for men
% in their 80’s who take Tai Chi lessons for 6 months is greater than zero. The P-value of this test is 0.004.
% Interpret this value.
The P-value of 0.004 means that if the null hypothesis is true (the true mean difference in flexibility rating is zero),
then there is a 0.4\% chance of getting a sample mean difference as extreme as the one we got (1.375) or more extreme.

\subsection*{d)}
% If the result of this study is statistically significant, can you conclude that the difference in the mean
% flexibility rating was caused by the Tai Chi lessons? Why or why not?
No, we cannot conclude that the difference in the mean flexibility rating was caused by the Tai Chi lessons because
the study is observational and thus there could be other factors that influenced the difference in flexibility rating.



\end{document}
