% arara: latexmk: { options: [ '-pdf' ] }
% arara: latexmk: { options: ['-c' ] }
\documentclass{article}
\usepackage{graphicx} % Required for inserting images
\usepackage{amsmath}


\title{AP Physics C Test 1 Corrections}
\author{Henry Beveridge}
\date{March 25, 2024}

\begin{document}

\maketitle

\section*{2.}
Constant acceleration means that we can solve this using the following equation:
\begin{align*}
    \theta_f&=\frac{1}{2}\alpha t^2 \\
    16 \cdot2\pi&= \frac{1}{2}\alpha(10)^2 \\
    \frac{32\pi\cdot 2}{100} &= \alpha \\
    \alpha &= \frac{16\pi}{25} \,\text{s}^{-2}
\end{align*}
Then we can use the following equation for $\omega_f$:
\begin{align*}
\alpha t &= \omega_f \\
\frac{16\pi}{25}\cdot10 &= \omega_f \\
\omega_f &= \frac{32\pi}{5} \,\text{s}^{-1}
\end{align*}

\section*{4.}
First calculate moment of inertia:
\begin{align*}
I &= I_\text{thin rod} + I_\text{displacement} \\
&= \frac{1}{12}ML^2 + Md^2 \\
&= \frac{1}{12}ML^2 + \frac{1}{16}ML^2 \\
&= \frac{7}{48}ML^2
\end{align*}
Then we use the following equations for torque to solve for $\alpha$:
\begin{align*}
    I\alpha &= \tau_\text{net} \\
    &= Mg\cdot \frac{L}{4} \cdot \text{sin(150)} \\
    \alpha &= \frac{MgL}{8I} \\
    &= \frac{6g}{7L}
\end{align*}

\section*{5.}
\textbf{a)} Calculate the mechanical energy for both carts and compare it to cart B.

\begin{align*}
    ME_A &= \frac{1}{2}Mv^2 + MR^2\omega^2 =\frac{3}{2}Mv^2\\
    &= 2Mgh\\
    gh_A &=\frac{3}{4}v^2 \\
    ME_B &= \frac{3}{2}Mv^2 \\
    &= 3Mgh \\
    gh_B &= \frac{1}{2}v^2 \\
    \frac{gh_B}{gh_A} &= \frac{2}{3}
\end{align*}
Thus, $h_B=\dfrac{2}{3}h$.

\vspace{1cm}

\noindent\textbf{b)} Friction must slow the rotation of the wheel, so it points up the ramp. Friction also is the only thing causing torque, so we can find $\alpha$ that way. Then we can just use equations for net torque and net force.  

\begin{align*}
    \tau_\text{net}&= MR^2\alpha = F_f R \\
    \alpha &= \frac{F_f}{MR} \\
    a &= \frac{F_f}{M} \\
    F_\text{net} &= 2Ma= 2Mg\cdot\text{sin}(\theta)-F_f \\
    3F_f&=2Mg\cdot\text{sin($\theta$)} \\
    F_f&=\frac{2}{3}Mg\cdot\text{sin($\theta$)}
\end{align*}

\end{document}
